% Metódy inžinierskej práce
\documentclass[12pt,a4paper]{article}
\usepackage[slovak]{babel}
\usepackage[utf8]{inputenc}
\usepackage[T1]{fontenc}
\usepackage{setspace}
\usepackage{geometry}
\geometry{margin=3cm}
\usepackage{titlesec}
\usepackage{lmodern}

\titleformat{\section}{\large\bfseries}{\thesection.}{0.8em}{}
\titleformat{\subsection}{\normalsize\bfseries}{\thesubsection.}{0.8em}{}

\setstretch{1.4}

\begin{document}

\begin{center}
    {\LARGE \textbf{Využitie umelej inteligencie v zdravotníctve pre urýchlenie administratívnych činností}}\\[0.6cm]
    {\large Projektová správa}\\[1cm]
\end{center}

\section{Úvod}

Umelá inteligencia je jednou z najvýraznejšie a najrýchlejšie sa rozvíjajúcich oblastí súčasnej éry. Nachádza si širokospektrálne využitie v spoločnosti. Integruje ľudskú skúsenosť a inteligenciu k tomu, aby napodobnila myslenie človeka tak, že dokáže brať do úvahy súčasne ohromné množstvo dát a údajov, vďaka čomu môže robiť presné odhady.

Každý človek chce byť zdravý a očakáva, že jeho fungovanie a zdravotný stav bude pod dostatočnou kontrolou. Aj z tohto dôvodu je uplatnenie umelej inteligencie v zdravotníctve hádam najočakávanejšou udalosťou súčasnej doby. Diagnostika a včasné odhalenie chorôb ako aj rýchla a účinná liečba založená na presných výpočtoch vychádzajúcich z veľkého množstva dát je nádejou pre budúcnosť. 

V našej práci sa však chceme zaoberať hlavne využitím umelej inteligencie na odbúranie administratívnej záťaže spojenej s úkonmi v zdravotníctve. Zníženie administratívnej záťaže vytvorí predpoklad na lepší prístup personálu k liečbe samotnej.

\section{Prínosy a výhody využitia AI}

Štúdia \textit{Managing Healthcare Using Artificial Intelligence} z roku 2025 uvádza, že využitie umelej inteligencie v administratívnych procesoch predstavuje zásadný krok ku digitalizácii a efektívnejšiemu fungovaniu zdravotníckych zariadení. Veľká časť pracovného času zdravotníckeho personálu, či už sestier alebo lekárov, predstavuje vypĺňanie rôznych formulárov, správ pre zdravotné poisťovne, zdravotnej dokumentácie pacientov, komunikáciu s laboratóriami a inými nemocničnými pracoviskami. 

Umelá inteligencia môže toto všetko automatizovať, a tým znížiť administratívnu záťaž lekárov, sestier a ďalšieho personálu a odbúrať chyby, ktorých sa môžu z dôvodu ľudského faktora dopustiť. Takto umelá inteligencia významne zvýši presnosť spracovania dát.

V systéme, ktorý navrhujú, je umelá inteligencia schopná analyzovať a kategorizovať zdravotné záznamy, spravovať poistné údaje pacientov a generovať podklady pre lekárske správy. Vďaka tomu sa zdravotnícky personál môže z dôvodu ušetreného času sústrediť na odborný a osobný prístup k pacientovi, čím sa zvýši čas venovaný priamo komunikácii.

Podľa štúdie \textit{Applications of AI in Healthcare Sector for Enhancement of Medical Decision Making and Quality of Service} z roku 2022 sa čoraz viac nemocníc a zdravotníckych spoločností obracia na AI chatboty, ktoré pacientom umožňujú rezervovať termíny, klásť otázky, vyhľadávať informácie o dostupnosti lekárov a dokonca spravovať údaje o predpisoch. Takéto riešenia šetria čas zdravotníckeho personálu, ktorý môže ušetrený čas využiť buď pre seba v rámci prestávky, alebo pre pacientov, ktorí nebudú musieť dlho čakať. Tým sa zlepšuje celková skúsenosť pacienta.

Okrem úspory času je dôležitý aj ekonomický aspekt. Použitím umelej inteligencie na vykonávanie administratívnych činností výrazne znížime prevádzkové náklady. Ušetrené financie môžu byť použité na liečebné účely.

Autori Avinash Kumar a Sujata Joshi uvádzajú konkrétne výhody implementácie umelej inteligencie v administratíve: jednoduchšie a rýchlejšie objednávanie, zníženie čakacej doby, kratší čas pacientov strávený v nemocnici a úspora nákladov. 

Macriga, Poorvaja a Sangeetha uvádzajú, že nasadenie umelej inteligencie v administratíve vedie k plynulejšiemu chodu zariadenia, rýchlejšiemu zdieľaniu informácií a lepšej koordinácii medzi oddeleniami.

Tieto výhody dokazujú, že umelá inteligencia môže byť významným nástrojom na optimalizáciu prevádzky nemocníc a kliník a zároveň pomôže poskytovať kvalitnejšiu starostlivosť s menšou administratívnou záťažou.

\section{Výzvy a riziká využitia AI}

Napriek rastúcemu použitiu umelej inteligencie v zdravotníctve existujú viaceré výzvy a riziká, ktoré treba pri jej zavádzaní riešiť. Štúdia \textit{Applications of AI in Healthcare Sector for Enhancement of Medical Decision Making and Quality of Service} uvádza, že zdravotníctvo generuje obrovské množstvo dát, ktoré sú často neúplné, nekonzistentné alebo zašumené. To sťažuje proces učenia sa algoritmov umelej inteligencie.

Ďalším nezanedbateľným problémom je ochrana súkromia a bezpečnosť osobných údajov pacientov. Pacienti sa obávajú, že citlivé informácie o nich a ich zdravotnom stave môžu byť zneužité alebo predané. Preto je potrebné, aby systémy umelej inteligencie boli v súlade s etickými princípmi a chránili osobné údaje a dáta.

V neposlednom rade hrozia aj kybernetické útoky, pri ktorých viac digitalizované zdravotné zariadenia môžu byť náchylnejšie na útoky z internetu, čo by znamenalo výpadok služieb.

Významným problémom je aj nízka úroveň digitalizačnej pripravenosti v niektorých zdravotníckych zariadeniach. Niektoré používajú rovnakú techniku ako pred dvadsiatimi rokmi. Zdravotnícki pracovníci predstavujú tiež výzvu, keďže množstvo staršieho personálu nemá skúsenosť s novou technikou. Musia pochopiť prínos umelej inteligencie, aby sa neobávali straty pracovných pozícií a zároveň sa naučili nový spôsob práce.

\section{Prípadové štúdie využitia AI v zdravotníctve}

Jedným z príkladov sú AI chatboty. Tieto rezervujú pacientom termíny, sumarizujú anamnézy a riešia poistné záležitosti. Projekty ako MFine, ADA Health či Babylon Health sa využívajú na komunikáciu s pacientmi. Vďaka nim získajú pacienti okamžité odpovede na svoje otázky. Keďže údaje o zdravotnom stave pacientov a symptómoch získajú lekári ešte pred návštevou, môžu sa lepšie pripraviť. Takto sa šetrí aj čas vyšetrenia pacienta aj čas zadávania dát. 

Podobný princíp sa používa aj v logistickom priemysle, kde operátor skladu využíva AI v podobe náhlavného zariadenia na identifikovanie uloženia balíkov, ich presun a kontrolu. V zdravotníctve by to mohlo byť zariadenie, do ktorého zdravotný personál diktuje anamnézu pacienta, pýta sa AI na radu alebo nechá preložiť cudzojazyčnú anamnézu pacienta.

Ďalším príkladom je Royal Bolton Hospital vo Veľkej Británii, ktorá zaviedla systém qXR na čítanie RTG snímok. Umelá inteligencia výrazne skrátila čas spracovania a popisu snímok. Toto pomáha pri rýchlejšom a presnejšom rozhodovaní lekárov a zlepšuje tok dát v nemocnici.

\section{Automatické rozpoznávanie reči v zdravotníctve}

Automatické rozpoznávanie reči (ASR -- \textit{Automatic Speech Recognition}) predstavuje jednu z významných výziev pri využívaní umelej inteligencie v medicíne a v administratívnych procesoch zdravotníctva. Táto technológia umožňuje v reálnom čase prepis hovoreného slova zo zaznamenaného audio signálu do písanej textovej podoby. V klinickej praxi sa ASR využíva predovšetkým pri tvorbe zdravotnej dokumentácie, diktovaní lekárskych správ, zaznamenávaní komunikácie medzi lekárom a pacientom, ako aj pri tvorbe prepúšťacích správ.

Podľa štúdie \textit{A Comprehensive Analysis of Speech Recognition Systems in Healthcare} sú systémy rozpoznávania reči schopné výrazne znížiť čas potrebný na vytváranie zdravotnej dokumentácie, čím prispievajú k plynulejšiemu priebehu vyšetrení a efektívnejšiemu fungovaniu nemocníc a ambulancií. Autori zároveň uvádzajú, že moderné ASR systémy založené na metódach hlbokého učenia dosahujú podstatne vyššiu presnosť v porovnaní so staršími prístupmi. Kľúčovým faktorom ich úspešnosti je trénovanie na doménovo špecifických dátach a dôsledné dodržiavanie presnej medicínskej terminológie.

\subsection{Deep learning a moderné modely pre prepis reči}

Rozvoj ASR systémov úzko súvisí s nástupom modelov hlbokého učenia, ako sú konvolučné neurónové siete, rekurentné neurónové siete a najnovšie transformerové architektúry. Metodologický prehľad publikovaný v štúdii \textit{Speech Technology for Healthcare: Opportunities, Challenges, and State of the Art} identifikuje deep learning ako kľúčový faktor, ktorý umožnil zásadný posun v presnosti rozpoznávania reči, a to najmä v kontexte medicínskej terminológie a náročného nemocničného prostredia. Tieto systémy musia efektívne zvládať prítomnosť šumu, rozdielne prízvuky hovoriacich a používanie vysoko špecializovanej terminológie.

Autori uvedenej štúdie zároveň poukazujú na skutočnosť, že reč predstavuje prirodzený a ľahko dostupný zdroj dát, ktorý je možné zbierať bez narušenia klinického pracovného postupu. Z tohto dôvodu sa ASR technológie stávajú vhodným základom pre ďalšie AI systémy, ktoré nadväzujú na textový prepis, ako sú systémy sumarizácie, automatického generovania správ alebo rozhodovacej podpory v klinickej praxi.

\section{Generatívna AI a automatická tvorba zdravotných záznamov}

Samotný prepis reči do textovej podoby predstavuje iba prvý krok v automatizácii klinickej dokumentácie. Súčasným trendom je prepájanie ASR systémov s generatívnymi modelmi umelej inteligencie, najmä veľkými jazykovými modelmi (LLM -- \textit{Large Language Models}), ktoré dokážu z neštruktúrovaného textu automaticky vytvárať štruktúrované zdravotné záznamy.

Štúdia \textit{A GenAI Framework for Medical Note Generation} predstavuje systém \textit{MediNotes}, ktorý kombinuje ASR, generatívne jazykové modely a prístup Retrieval-Augmented Generation (RAG) na automatickú tvorbu SOAP poznámok z rozhovorov medzi lekárom a pacientom. Výsledky ukazujú, že takýto prístup výrazne znižuje administratívnu záťaž lekárov, šetrí čas potrebný na dokumentáciu a zároveň zlepšuje celistvosť, presnosť a čitateľnosť zdravotných záznamov.

Podobné závery prináša aj riešenie \textit{DocGenie}, ktoré využíva pokročilé ASR modely v kombinácii s veľkými jazykovými modelmi, ako je napríklad GPT-4, na generovanie kompletných lekárskych správ z klinických rozhovorov. Autori poukazujú najmä na výrazné skrátenie času potrebného na tvorbu dokumentácie a zníženie chýb spôsobených manuálnym zapisovaním údajov.

\section{Veľké jazykové modely v zdravotníctve}

Veľké jazykové modely (LLM -- Large Language Models) sa rýchlo stávajú transformačným nástrojom v oblasti spracovania klinických textov a neštruktúrovaných dát v zdravotníctve, vrátane klinických poznámok, rozhovorov medzi lekárom a pacientom či elektronických zdravotných záznamov. LLM dokážu nielen generovať text, ale aj analyticky spracovávať kontext, extrahovať kľúčové informácie a sumarizovať obsah, čo má potenciál významne podporovať klinické rozhodovanie, dokumentáciu a rýchlejšiu orientáciu v komplexných dátach. Takéto modely môžu asistovať pri spracovaní veľkého množstva neštruktúrovaných textových záznamov, odpovedať na odborné otázky či sumarizovať lekárske poznámky, čím zvyšujú efektivitu a znižujú administratívnu záťaž zdravotníckeho personálu. Zároveň však odborná literatúra zdôrazňuje potrebu dôkladnej validácie modelov a odborného dohľadu, keďže výstupy generované LLM nemusia vždy presne odrážať klinickú realitu a môžu obsahovať nepresnosti.

Autori prehľadových štúdií zároveň upozorňujú na viaceré výzvy spojené s nasadením LLM v klinickej praxi, medzi ktoré patrí ochrana osobných zdravotných údajov, algoritmická zaujatost a problematika integrácie do existujúcich nemocničných informačných systémov a elektronických zdravotných záznamov. Nedostatočné zabezpečenie dát alebo slabé riadenie prístupových práv môže viesť k narušeniu dôvernosti citlivých informácií. Okrem toho modely trénované na nevyvážených alebo neúplných dátach môžu reprodukovať existujúce zaujatosti, čo môže viesť k nespravodlivému alebo nepresnému hodnoteniu pacientov. Úspešná implementácia LLM riešení si preto vyžaduje starostlivé plánovanie interoperabilných rozhraní, súlad s platnou legislatívou a dôraz na bezpečnosť a spoľahlivosť systémov.

\section{Riziká a obmedzenia AI transkripcie}

Napriek významným prínosom technológií automatického rozpoznávania reči (ASR -- Automatic Speech Recognition) a generatívnej umelej inteligencie v oblasti klinickej dokumentácie existuje viacero rizík spojených s ich praktickým využitím. Prehľadové štúdie poukazujú na problémy spôsobené nízkou kvalitou vstupných dát, hlukom v prostredí, rozdielnymi prízvukmi hovoriacich alebo nejednoznačnosťou medicínskej terminológie. Nesprávna transkripcia odborných pojmov, názvov liekov alebo diagnóz môže viesť k chybám v zdravotnej dokumentácii a následne aj k nesprávnym klinickým rozhodnutiam, čo v extrémnych prípadoch môže mať vážne až fatálne dôsledky pre pacienta.

Ďalším zásadným problémom je ochrana citlivých zdravotných údajov, keďže hlasové nahrávky a ich textové prepisy predstavujú osobné identifikovateľné údaje. Integrácia AI transkripčných systémov do zdravotníckych pracovných tokov musí byť v súlade s platnými legislatívnymi požiadavkami, ako je napríklad nariadenie GDPR, a s etickými princípmi ochrany súkromia pacientov. Odborné štúdie zdôrazňujú potrebu implementácie technických opatrení, ako sú šifrovanie, riadenie prístupových práv a auditovateľnosť systémov, ako aj potrebu zapojenia zdravotníckych odborníkov do kontroly a overovania výstupov AI systémov.

\section{Záver}

Umelá inteligencia už zásadne mení a v blízkej budúcnosti úplne zmení spôsob fungovania zdravotníctva. Už dnes existuje mnoho spôsobov jej využitia v praxi. Najväčšou výzvou pre náš projekt je integrácia všetkých údajov a administratívnych úkonov. Takáto integrácia by predstavovala komplexné riešenie pre zdravotníctvo – znížila by administratívnu záťaž, urýchlila prácu s pacientom a znížila finančné náklady.

\end{document}
